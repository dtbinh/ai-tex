\documentclass[b5paper,11pt]{book}

\usepackage{geometry} 					% поля страницы

\usepackage{cmap}                       % Поддержка поиска русских слов в PDF (pdflatex)
\usepackage[T2A]{fontenc}				% Поддержка русских букв
\usepackage[utf8]{inputenc}            	% Выбор языка и кодировки
\usepackage[english, russian]{babel}	% Языки: русский, английский

\usepackage[unicode]{hyperref}			% Русский язык для оглавления pdf
\usepackage{bookmark}					% Оглавление в pdf
\usepackage{graphicx} 					% Подключаем пакет работы с графикой

\usepackage{amsmath,amssymb}

\graphicspath{{../../images/}} 			% Пути к изображениям

\geometry{left=2cm,right=2cm,top=2cm,bottom=2cm}	% Геомтерия страницы

\usepackage[
%	autolang=hyphen,
language=auto,
autolang=other,
backend=biber,
style=gost-numeric
]{biblatex}
\addbibresource{ids.bib}

\DeclareSourcemap{
	\maps[datatype=bibtex, overwrite]{
		\map{
			\step[fieldset=langid, fieldvalue=english]
			\step[fieldset=doi, null]
			\step[fieldset=issn, null]
			\step[fieldset=isbn, null]
			\step[fieldset=url, null]
			\step[fieldsource=language, fieldset=langid, origfieldval]
		}
	}
}

\newtheorem{Def}{Определение}
\newtheorem{Th}{Теорема}

\let\cleardoublepage\clearpage

\begin{document}
	\begin{titlepage}
		\begin{center}
			{\bfseries  Федеральное государственное автономное \\
				образовательное учреждение высшего образования\\
				<<Российский университет дружбы народов>>
				
			}

			\vspace{-5pt}
			\noindent\rule{\textwidth}{2pt}
			
			\vspace{50pt}
			{\Large\bfseries А.\,И.~Панов}
			
			\vspace{100pt}
			{\Huge\bfseries Интеллектуальные динамические системы}
			
			\vspace{20pt}
			{\Large\itshape Учебно-методическое пособие}
			
			\vfill
			{\bfseries Москва\\
				Российский университет дружбы народов\\
				2015
			}
		\end{center}
	\end{titlepage}
	
	\chapter*{}
	
	В пособии рассмотрены основные методы, применяющиеся при построении интеллектуальных динамических систем (ИДС). Одним из основных свойств ИДС является свойство иерархичности, уровневости организации всех процессов, связанных с ИДС, начиная от управления такими системами и заканчивая процессами самоорганизации в их базе знаний.
	
	
	\tableofcontents %% содержание
		
	\chapter*{Введение}
	\addcontentsline{toc}{chapter}{Введение}
	Динамические интеллектуальные системы "--- результат интеграции интеллектуальных систем с динамическими системами. В общем случае это двухуровневые динамические модели, где один из уровней отвечает за стратегию поведения системы (или, как иногда говорят, носит делиберативный характер), а другой уровень отвечает за реализацию конкретной (в том числе, математической) модели.
	
	К таким системам относятся сложные естественные системы, такие как экологические, социальные и политические системы, а также такие динамические системы, в которых зависимости настолько сложны, что не допускают своего обычного аналитического представления. Сложность задач управления, в которых существенная роль принадлежит экспертным суждениям и знаниям человека, заставляет в дополнение к количественным методам или вместо них применять такие подходы, в которых в качестве значений переменных допускаются не только числа, но и слова или предложения искусственного или естественного языка. 
	
	Потребность в моделях такого рода назрела в связи с развитием, например, беспилотных средств транспортного и иного назначения. В частности, в беспилотных автономных самолетах и вертолётах одним из уровней управления должен являться делиберативный уровень управления, решающий задачи, например, планирования полёта или выбора траектории или выбора цели. Другой уровень управления "--- назовем его активным "--- реализует требуемые действия. Например, на  делиберативном уровне управления беспилотным вертолётом принимается решение о зависании над целью, тогда на активном уровне начинает работать математическая модель зависания, вырабатывающая требуемые управления для исполнительных механизмов.
	
	\chapter{Представление статических знаний}
	
	\section{Логика предикатов первого порядка}
	Одним из наиболее изученных формальных языков является язык исчисления предикатов первого порядка. Существуют работы, где язык исчисления предикатов рассматривается как язык представления знаний, однако, это не главное его назначение и мы будем использовать его, главным образом, в качестве средства описания элементов конструкций других языков, более ориентированных на представление знаний. 
	
	Опишем вначале основные конструкции языка исчисления предикатов первого порядка и их интерпретацию в духе \cite{Klini1973,Keisler1977}.
	
	\subsection{Описание языка}
	Основные конструкции языка $L$ – языка исчисления предикатов первого порядка называются формулами. Введем вначале \textit{алфавит} языка $L$. Алфавит включает:
	\begin{enumerate}
		\item Счетное множество букв: $z,y,x,\dots$, которое будем называть множеством символов для обозначения переменных языка.
		\item Счетное множество букв $a,b,c,\dots$, которое будем называть множеством символов для обозначения констант языка.
		\item Счетное множество прописных букв $P,Q,\dots$ для обозначения предикатных символов языка.
		\item Счетное множество строчных букв $f,g,\dots$ для обозначения функциональных символов.
		\item Символы для логических связок $\rightarrow$ (влечет), $\neg$ (не).
		\item Символ для квантора $\forall$ (для любого);
		\item (,) "--- скобки.
	\end{enumerate}
	
	Предикатные буквы $P,Q,\dots$ и функциональные буквы $f,g,\dots$ могут быть $n$-местными или, как еще говорят, $n$-арными. Иначе говоря, с каждым предикатным или функциональным символом будем связывать некоторое натуральное число, равное числу его аргументов.
	
	Определим теперь понятие формулы или правильно построенного выражения языка исчисления предикатов первого порядка. \textit{Формулы} языка определяются индуктивным образом. Начнем с определения \textit{терма} языка:
	\begin{enumerate}
		\item Переменная есть терм.
		\item Константа есть терм.
		\item Если $t_1,t_2,\dots,t_m,\dots,t_n$ "--- термы, а $f$ и $g$ "--- функциональные символы арности $m$ и $n$, соответственно, то  $f(t_1,t_2,\dots,t_m)$ и $g(t_1,t_2,\dots,t_n)$ также термы.
		\item Если $t_1,t_2,\dots,t_m,\dots,t_n$ "--- термы, а $P$ и $Q$ "--- предикатные символы арности $m$ и $n$, соответственно, то $P(t_1,t_2,\dots,t_m)$ и $Q(t_1,t_2,\dots,t_n)$ "--- атомарные формулы.
		\item Атомарная формула есть формула.
		\item Если $A,B$ "--- формулы, то $(A\rightarrow B)$, $\neg A$, $\neg B$ "--- формулы.
		\item Если $A$ "--- формула, то $\forall x A$ "--- формула.
		\item Всякое слово в алфавите языка является формулой тогда и только тогда, когда это можно показать с помощью конечного числа применений п.п. 1-7.
	\end{enumerate}

	Таким образом, мы завершили одно из возможных определений языка исчисления предикатов первого порядка. Существуют и другие определения, однако, язык, определенный нами, является полным, т.~е. в нем выразимо все то, что выразимо в языках (исчисления предикатов первого порядка), определенных любым иным способом.
	
	Можно, например, определить логические связки $\wedge, \vee$(читается \textit{и} и \textit{или}), выразив их через связки $\rightarrow$ и $\neg$:
	\begin{itemize}
		\item $A\wedge B = \neg(A\rightarrow\neg B)$,
		\item $A\vee B =\neg A\rightarrow B$.
	\end{itemize}
	
	Квантор существования "--- $\exists$ (существует) также выражается через квантор всеобщности и отрицание: $\exists x A(x) = \neg\forall x\neg A(x)$.
	
	Разумеется, $\wedge$, $\vee$ и $\exists$ с тем же успехом можно было бы включить в язык в качестве трех дополнительных символов. Есть, однако, некоторые преимущества в том, чтобы сохранить список символов как можно более коротким. Например, индуктивные определения и доказательства по индукции оказываются в этом случае короче.
	
	В дальнейшем нам придется использовать понятия \textit{свободного} и \textit{связанного} вхождения переменной в формулу. Вхождение переменной $x$ в формулу $A$ называется связанным, если эта переменная следует за квантором существования или всеобщности, предшествующими  формуле $A$. В противном случае, вхождение переменной называется свободным. Если в формуле $A$ отсутствуют свободно входящие в нее переменные (т.~е. либо все переменные связаны, либо просто отсутствуют), то формула называется \textit{замкнутой формулой} или \textit{предложением}. Атомарную замкнутую формулу будем называть \textit{фактом}. В том случае, если язык состоит только лишь из предложений, то он называется пропозициональным языком, а буквы $A,B,\dots$, входящие в формулы этого языка "--- пропозициональными переменными.
	
	\subsection{Основные понятия исчисления}
	Рассмотрим вкратце основные понятия исчисления предикатов первого порядка.
	
	Введем вначале аксиомы исчисления предикатов:
	\begin{enumerate}
		\item\label{ax1} $A\rightarrow (B\rightarrow A)$,
		\item\label{ax2} $(A\rightarrow (B\rightarrow C))\rightarrow ((A\rightarrow B)\rightarrow(A\rightarrow C))$,
		\item\label{ax3} $(\neg A\rightarrow \neg B)\rightarrow(B\rightarrow A)$.
	\end{enumerate}  
	А затем правила вывода:
	\begin{description}
		\item[Правило отделения:] если выводимо $A$ и выводимо $A\rightarrow B$, то выводимо $B$.
		\item[Правило подстановки:] в любую аксиому на место любой пропозициональной переменной можно подставить любое предложение, предварительно переименовав пропозициональные переменные подставляемого предложения так, чтобы они не совпадали с пропозициональными переменными аксиомы.
	\end{description}
	
	Если в аксиомах \ref{ax1} -- \ref{ax3} все переменные являются пропозициональными, то такое исчисление называется \textit{пропозициональным исчислением} или \textit{исчислением высказываний}.
	
	Рассмотрим пример вывода в исчислении высказываний. Возьмем, например, три закона логики, сформулированные Аристотелем и называемые постулатами Аристотеля. В языке исчисления высказываний они записываются следующим образом: 
	Пусть $P$ "--- пропозициональная переменная исчисления высказываний.
	\begin{description}
		\item[Постулат 1 (закон тождества):] $P\rightarrow P$.
		\item[Постулат 2 (закон исключения третьего):] $P\vee\neg P$.
		\item[Постулат 3 (закон противоречия):] $\neg(P\wedge\neg P)$.
	\end{description}
	
	Докажем один из постулатов, например закон тождества. Используем  аксиому \ref{ax1} и правило подстановки (вместо $B$ подставим $P\rightarrow P$) и получим $A\rightarrow((P\rightarrow P)\rightarrow A)$. Из аксиомы \ref{ax2}:
	\[
		(A\rightarrow((P\rightarrow P)\rightarrow C))\rightarrow((A\rightarrow(P\rightarrow P))\rightarrow(A\rightarrow C)).
	\]
	Теперь вместо $A$ и $C$ подставим $P$:  
	\[
		\underbrace{(P\rightarrow((P\rightarrow P)\rightarrow P))}_{X}\rightarrow\underbrace{((P\rightarrow(P\rightarrow P))\rightarrow(P\rightarrow P))}_{Y}.
	\]
	Затем применим правило отделения: та часть последней формулы, которая обозначена через $X$ является аксиомой, т.~е. выводима, тогда в силу правила отделения, выводима формула, обозначенная через $Y$. Теперь применим правило отделения к $Y$: 
	\[
		\underbrace{(P\rightarrow(P\rightarrow P))}_{X'}\underbrace{\rightarrow(P\rightarrow P)}_{Y'}.
	\]
	и, рассуждая таким же образом, получим, что $Y’$ "--- выводимо. Таким образом, закон тождества Аристотеля является \textit{теоремой} исчисления высказываний. Действуя таким же образом, можно доказать, что второй и третий постулаты Аристотеля также являются \textit{теоремами} исчисления высказываний.
	
	Однако, исчисление предикатов первого порядка не исчерпывается приведенными выше тремя аксиомами и правилами вывода. Смысл кванторов устанавливается еще двумя аксиомами и одним правилом вывода.
	\begin{enumerate}
		\setcounter{enumi}{4}
		\item\label{ax4} $\forall x((A\rightarrow B)\rightarrow(A\rightarrow\forall x B))$, где $x$ не является свободной переменной в $A$;
		\item\label{ax5} $\forall t A(t)\rightarrow A(x)$, где $t$ "--- терм, а $x$ не содержится в $t$ в качестве свободной переменной.
	\end{enumerate}
	Четвертая аксиома называется аксиомой генерализации, а вторая "--- аксиомой спецификации.
	
	\begin{description}
		\item[Правило обобщения:] $A\rightarrow\forall xA$, где $x$ "--- свободная переменная в $A$.
	\end{description}

	Аксиомы \ref{ax1}--\ref{ax5} исчисления предикатов первого порядка (или математической логики первого порядка) называются \textit{логическими} аксиомами, они описывают логические законы, справедливые всегда, независимо от предметной области. Если же к аксиомам \ref{ax1}--\ref{ax5} добавить еще и аксиомы, описывающие некоторую предметную область, например, арифметику или теорию групп, то получим \textit{формальную теорию} "---  формальную арифметику или формальную теорию групп, соответственно. При этом, разумеется, в алфавит языка следует ввести специальные функциональные символы, такие как сложение в арифметике или умножение в теории групп.
	
	Словосочетание <<первый порядок>> относится исключительно к тому обстоятельству, что кванторы $\forall$ и $\exists$ действуют на некотором универсальном множестве $U$. Логика второго порядка разрешает одному из кванторов действовать на подмножествах множества $U$ и на функциях из степеней $U$ в $U$. Логика третьего порядка может использовать кванторы по множествам функций и т.\,д. Уже из этих разъяснений видно, что в логиках более высоких порядков (как говорят, более сильных логиках) используются и некоторые нелогические понятия, такие как множество.
	
	\subsection{Формальные системы и интпретация}
	Будем полагать, что если заданы некоторый алфавит, множество формул, множества аксиом  и правил вывода, тотем самым задана некоторая формальная система. Иначе говоря, формальная система $F$ представляет собой совокупность следующих объектов:
	\[
		F=\langle T, Р, А, \Pi\rangle,
	\]
	где $T$ "--- конечное множество символов; $P$ "--- множество правил грамматики, применение которых к символам из $T$, позволяет строить правильно построенные формулы; $А$ "--- множество аксиом;	$\Pi$ "--- множество правил вывода. 
	
	Если среди аксиом имеются нелогические аксиомы (аксиомы, описывающие некоторую предметную область), то формальная система называется \textit{формальной теорией}.
	\begin{Def}
		\textit{Выводом} (или \textit{доказательством}) в формальной системе называется конечная последовательность правильно построенных формул 
		\[
			A_1,A_2,\dots,A_n,
		\] таких что каждая из формул последовательности либо является аксиомой либо получена из предыдущих формул последовательности с использованием аксиом и правил вывода.
	\end{Def}
	
	Формула $A_n$ в этом случае называется \textit{выводимой формулой} (или \textit{теоремой}) формальной системы $F$.
	
	\section{Атрибутивная логика}
	\section{Семантические сети}
	


	\chapter{Представление процедурных знаний}
	
	\section{Системы правил}
	\section{Семиотическое представление}
	


	\chapter{Пополнение знаний}
	
	\section{Проблема привязки символов}
	\section{Биологически правдоподобные методы}
	\section{Выявление причинно-следственных связей}
	
	
	
	\chapter{Планирование поведения}
	
	\section{Классические алгоритмы планирования}
	\subsection{Планирование как доказательство теорем}
	\subsection{Планирование в пространстве состояний}
	\subsection{Планирование на основе прецедентов}
	
	\section{Планирование с удовлетворением ограничений}
	\section{Графические системы планирования}
		
		
	
	\chapter{Системы, основанные на правилах}
	
	\section{Состояния и траектории}
	\section{Синтез управления}
	\section{Синтез обратной связи}
	\section{Основы теории управляемости}
	


	\chapter{Практические задания в системе Jadex}
	\section{Задачи с международного соревнования планировщиков}
	Ниже представлен список задач с одного из треков одной из самых главных конференций по планированию "--- ICAPS за 2014г. Представленный трек из программы International Planning Competition 2014 включает в себя задачи по детерминированному планированию.
	\subsection{Бармен}
	Автор "--- Sergio Jiménez Celorrio.
	
	Представим себе робота-бармена, который орудует дозаторами, стаканами и шейкером. Цель планировщика "--- построить план действий робота по приготовлению необходимого количества коктейлей. Необходимо учесть, что манипуляторы робота могут брать только один предмет за раз, а стаканы должны быть пустыми и чистыми, прежде чем их начинать заполнять.
	\subsection{Дайвинг в пещерах}
	Авторы: Nathan Robinson,Christian Muise, and Charles Gretton.
	
	Представим себе группу дайверов, каждый из которых может переносить по 4 баллона с воздухом. Необходимо нанять этих дайверов для спуска в подводную пещеру. У них стоит задача фотосъемки либо задача доставки полных баллонов воздуха для подготовки спуска других дайверов. Пещера слишком узкая, чтобы пропустить более одного дайвера за раз.
	
	Пещера является разветвленной и может быть представлена в виде ненаправленного ациклического графа. У всех дайверов единственная точка входа. Определенные конечные точки ответвлений пещеры являются целями для фотографирования. Как задача фотосъемки, так и обычного плавания расходуют воздух из баллонов. В конце дайверы должны покинуть пещеру и подняться на поверхность. Следовательно, они могут сделать только один спуск в пещеру.
	
	Некоторые дайверы не уверены в других и будут отказываться работать, если кто-то из них не работал прежде со своим коллегой. Стоимость оплаты труда дайвера обратно пропорциональна количеству времени, которое они тратят на работу. 
	\subsection{Детская закуска}
	Авторы: Raquel Fuentetaja, Tomás de la Rosa Turbides. 
	
	Задача состоит в том, чтобы приготовить и подать бутерброды группе детей, у некоторых из которых аллергия на глютен. Есть два действия по приготовлению бутербродов из их ингредиентов. Первое из них готовит один бутерброд, а второй делает то же, но с учетом того, что все ингредиенты должны быть без глютена. Есть также действия положить один бутерброд и подать несколько бутербродов. 
	
	В~начальных условиях даны ингредиенты для приготовления бутербродов. Цели заключаются в обслуживании детей бутербродами, к которым у них нет аллергии.
	\section{Внешняя среда и типы агентов}
	\section{Задание состояний}
	\section{Задание правил и стратегий}
	\section{Планирование поведения}
	\section{Задачи по планированию}
	
	

	\chapter*{Заключение}
	\addcontentsline{toc}{chapter}{Заключение}
	Немного о итогах курса
	\printbibliography
\end{document}