\documentclass[b5paper,11pt]{book}

\usepackage{geometry} 					% поля страницы

\usepackage{cmap}                       % Поддержка поиска русских слов в PDF (pdflatex)
\usepackage[T2A]{fontenc}				% Поддержка русских букв
\usepackage[utf8]{inputenc}            	% Выбор языка и кодировки
\usepackage[english, russian]{babel}	% Языки: русский, английский

\usepackage[unicode]{hyperref}			% Русский язык для оглавления pdf
\usepackage{bookmark}					% Оглавление в pdf
\usepackage{graphicx} 					% Подключаем пакет работы с графикой

\usepackage{amsmath,amssymb}

\graphicspath{{../../images/}} 			% Пути к изображениям

\geometry{left=2cm,right=2cm,top=2cm,bottom=2cm}	% Геомтерия страницы

\usepackage[
%	autolang=hyphen,
language=auto,
autolang=other,
backend=biber,
style=gost-numeric
]{biblatex}
\addbibresource{tbi.bib}

\DeclareSourcemap{
	\maps[datatype=bibtex, overwrite]{
		\map{
			\step[fieldset=langid, fieldvalue=english]
			\step[fieldset=doi, null]
			\step[fieldset=issn, null]
			\step[fieldset=isbn, null]
			\step[fieldset=url, null]
			\step[fieldsource=language, fieldset=langid, origfieldval]
		}
	}
}

\let\cleardoublepage\clearpage

\begin{document}
	\begin{titlepage}
		\begin{center}
			{\bfseries  Федеральное государственное автономное \\
				образовательное учреждение высшего образования\\
				<<Российский университет дружбы народов>>
				
			}

			\vspace{-5pt}
			\noindent\rule{\textwidth}{2pt}
			
			\vspace{50pt}
			{\Large\bfseries А.\,И.~Панов}
			
			\vspace{100pt}
			{\Huge\bfseries Теоретические основы информатики}
			
			\vspace{20pt}
			{\Large\itshape Учебно-методическое пособие}
			
			\vfill
			{\bfseries Москва\\
				Российский университет дружбы народов\\
				2015
			}
		\end{center}
	\end{titlepage}
	
	\chapter*{}
	
	В пособии рассмотрены основные понятия теоретических основ информатики. Информатика (или, как принято называть ее за рубежом, компьютерные науки или Сomputer Science) является в настоящее время набором большого количества дисциплин, которые не смотря на свою разнородность как по применяемым методам и подходам, так и по методологии исследований, обладают тем не менее общими теоретическими основами. Центральной идеей в представленном курсе является рассмотрение вопросов, связанных с обработкой, хранением, передачей и приобретением информации, с точки зрения знаковых систем. Данная идея позволяет ввести многие методы и задачи, возникающие и в теории измерения количества информации, и в теории кодирования, и в теории программирования, с единых позиций и построить общую картину теоретических основ компьютерных наук.
	
	Особое внимание уделено рассмотрению основ в такой области информатики и кибернетики как искусственный интеллект, что, по мнению автора, обусловлено текущими запросами общества на всеобщую интеллектуализацию операций и более понятные и близкие человеку машинные интерфейсы. 
	
	Пособие не требует от читателя глубоких познаний в области информатики и искусственного интеллекта и предназначено для будущих специалистов в области компьютерных наук (Computer Science).
	
	
	\tableofcontents %% содержание
		
	\chapter*{Введение}
	\addcontentsline{toc}{chapter}{Введение}
	Теоретические основы информатики "--- это область информатики (компьютерных наук, Computer Science), которая рассматривает теоретические вопросы, связанные с получением, обработкой, хранением, передачей и преобразованием информации. Многие задачи, рассматриваемые в этой области опираются на определения и методы таких разделов математики,~как теория вероятности и математическая статистика. Все попытки дать определение общие информатики и информации сталкиваются с проблемой замкнутого цикла, который отличным литературным языком описан Станиславом Лемом \cite{Lem2015}. В этой связи во многих курсах теоретических основ информатики предпринимается попытка построения общего представления об информации с помощью конкретных примеров ее измерения,~получения и т.~п., т.~е. даётся эксплицитное определение на примерах использования. 
	
	В данном пособии используется иной подход, связанные с рассмотрением вопросов связанных с использованием в качестве носителя информации \textit{знаковых систем}. Таким образом, вместо определения информации, мы будем вводить понятия знака и системы, с помощью которых будут определены и основные операции с информацией, сводящиеся к операциям со знаковыми системами.
	
	
	
	\chapter{Теория информации}
	
	\section{Информация по Харлти и Шеннону}
	\section{Информация по Колмогорову}
	\section{Рекурсия и количество информации}
	
	
	\chapter{Представление данных}
	В общем о данных
	
	\section{Графы}
	
	\section{Сети}
	
	\section{Операции над данными}
	
	\chapter{Представление знаний}
	В общем о знаниях
	
	\section{Теория автоматов}
	\section{Формальные грамматики}
	\section{Формальные языки и системы}
	\section{Теорема Гёделя}
	
	\chapter{Теория алгоритмов}
	
	\section{Простейшие алгоритмы}
	\section{Основы теории сложности}
	\section{Алгоритмы на строках}
	\section{Алгоритмы на графах}
	\section{Алгоритмы в экономике}
	
	
	\chapter*{Заключение}
	\addcontentsline{toc}{chapter}{Заключение}
	Немного о целях курса
	\printbibliography
\end{document}