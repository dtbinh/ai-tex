\documentclass[a4paper,12pt]{article}

\usepackage{geometry} 					% поля страницы

\usepackage{cmap}                       % Поддержка поиска русских слов в PDF (pdflatex)
\usepackage[T2A]{fontenc}				% Поддержка русских букв
\usepackage[utf8]{inputenc}            	% Выбор языка и кодировки
\usepackage[english, russian]{babel}	% Языки: русский, английский
\usepackage{csquotes}

\usepackage[unicode]{hyperref}			% Русский язык для оглавления pdf
\usepackage{bookmark}					% Оглавление в pdf
\usepackage{graphicx} 					% Подключаем пакет работы с графикой
\usepackage{textgreek}					% Греческий текст без переключения в math-mode

\usepackage{amsmath,amssymb}
\usepackage[plain]{algorithm}
\usepackage[noend]{algpseudocode}

\usepackage[affil-it]{authblk}			% Красивая аффиляция авторов

\geometry{left=3cm,right=2cm,top=2cm,bottom=2cm}	% Геомтерия страницы
\graphicspath{{../../images/}} 			% Пути к изображениям

\usepackage[
%	autolang=hyphen,
language=auto,
autolang=other,
backend=biber,
style=gost-numeric
]{biblatex}
\addbibresource{../../biblio/library.bib}

\DeclareSourcemap{
	\maps[datatype=bibtex, overwrite]{
		\map{
			\step[fieldset=langid, fieldvalue=english]
			\step[fieldset=doi, null]
			\step[fieldset=issn, null]
			\step[fieldset=isbn, null]
			\step[fieldset=url, null]
			\step[fieldsource=language, fieldset=langid, origfieldval]
		}
	}
}

\begin{document}
	\title{Знак как элемент картины мира}
	\author{А.\,И.~Панов}
	\affil{Институт системного анализа РАН}
	
	\maketitle{}
	
	% оформление аннотации
	\begin{abstract}
		Главная идея
	\end{abstract}
	
	\section*{Введение}
	Основная цель работы "--- формализовать понятие знака и построить алгоритм его образования.
	
	\section{Понятие знака в науке}
	Понятие знака вводится и используются во многих отраслях знания. В первую очередь необходимо отметить, что знакам посвящено отдельное научное направление под названием \textit{семиотика} (от др.-греч. \textsigma\texteta\textmu\textepsilon\textiota\textomikron\textnu ~--- <<знак, признак>>), берущее своё начало в конце XIX~в. с работ американского философа Чарльза Пирса (Charles Sanders Peirce, 1839--1914) и французского лингвиста Фердинанда де Соссюра (Ferdinand de Saussure, 1857--1913). Семиотика является междисциплинарным направлением, предоставляющим общую терминологическую базу для исследователей из разных предметных областей. 
	
	Семиотику можно рассматривать некоторым исходным базисом, на основе которого в XX~в. стали развиваться новые научные направления уже в рамках конкретных предметных областей. Наиболее органично понятие знака было включено в лингвистику и языкознание, где принято говорить о языковом знаке как части некоторой языковой системы. В след за Соссюром центральное место знаку в структуре языка отводилось Луи Ельмслевом (Louis Trolle Hjelmslev, 1899-1965), Томасом Себеоком (Thomas Albert Sebeok, 1920-2001).
	
	Философскую ветвь семиотики в первую очередь принято связывать с Готтлобом Фреге (Friedrich Ludwig Gottlob Frege, 1848-1925), для которого понятие знака служило для разграничения смысла и значения, что в свою очередь имело большое значение для развития логики. Дальнейшее развитие философских аспектов знака получило в работах Чарльза Морриса (Charles William Morris, 1901-1979), Ролана Барта (Roland Barthes, 1915-1980).
	
	Отдельно следует отметить научное исследование знака и его роли в культуре и литературе. Литературоведческая ветвь семиотики пополнялась работами Альгирдаса Греймеса (Algirdas Julius Greimas, 1917-1992), Умберто Эко (Umberto Eco, род. 1932). Большой вклад в этом направлении был сделан отечественными литературоведами Романом Осиповичем Якобсоном (1896-1982) и Юрием Михайловичем Лотманом (1922-1993).
	
	Связь семиотики и психологии была впервые наглядно продемонстрирована в культурно~--историческом подходе Льва Семёновича Выготского (1896-1934). Психология выработало своё представление о структуре и роли знака в деятельности человека, достаточно сильно отличающееся от семиотических представлений. Дальнейшее развитие психологическая ветвь семиотики получила в работах Алексея Николаевича Леонтьева (1903-1979), Петра Яковлевича Гальперина (1902-1988).
	
	Связью семиотики, информационных и технических наук посвящена так называемая прикладная ветвь семиотики: компьютерная или прикладная семиотика. Это направление получило своё начало в работах отечественных специалистов по искусственному интеллекту Дмитрия Александровича Поспелова (род. 1932) и Геннадия Семёновича Осипова (род. 1948). Семиотическая парадигма в вопросах управления и представления знаний является новым и достаточно многообещающим подходом, в том числе направленным и на создание конкретных прикладных программных и технических систем.
	
	\subsection{Первые семиотические работы}
	
	Как отмечает Лотман, ещё в конце XVII~в. английский философ~--материалист Джон Локк (John Locke, 1632-1704) достаточно точно,~с точки зрения современных представлений, определил сущность и <<объём>> семиотики \cite[8]{Lotman2000}. По Локку задача семиотики "--- <<рассмотреть природу знаков, которыми ум пользуется для понимания вещей или для передачи своего знания другим>>. Однако, несмотря на столь раннее введение понятия знака в научную лексику, семиотические идеи ещё почти двести лет не получали своего развития.
	
	Отчёт научного направления под названием семиотика ведут с работ Пирса, чьи идеи получили широкое распространение только в 1930-е годы с опубликованием его архивов (русскоязычные переводы \cite{Pierce2000a,Pierce2000b, Pierce2009}). По
	
	Пирс создал первую классификацию знаков
	
	
	\subsection{Лингвистические работы}
	
	\subsection{Философские исследования}
	
	\subsection{Прикладная семиотика}
	
	\subsection{Формализация понятия}
	
	\section{Структура знака}
	\section*{Заключение}
	
   \printbibliography
\end{document} 