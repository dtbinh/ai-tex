\documentclass[11pt]{memoir}

\usepackage{geometry} 					% поля страницы

\usepackage{cmap}                       % Поддержка поиска русских слов в PDF (pdflatex)
\usepackage[T2A]{fontenc}				% Поддержка русских букв
\usepackage[utf8]{inputenc}            	% Выбор языка и кодировки
\usepackage[english, russian]{babel}	% Языки: русский, английский

\usepackage[unicode]{hyperref}			% Русский язык для оглавления pdf
\usepackage{bookmark}					% Оглавление в pdf
\usepackage{graphicx} 					% Подключаем пакет работы с графикой
\usepackage{memhfixc}

\usepackage{amsmath,amssymb}

\graphicspath{{../../images/}} 			% Пути к изображениям

\geometry{left=2cm,right=2cm,top=2cm,bottom=2cm}	% Геомтерия страницы

\usepackage[
	%	autolang=hyphen,
	language=auto,
	autolang=other,
	backend=biber,
	style=gost-numeric
]{biblatex}
\addbibresource{cmnn.bib}

\DeclareSourcemap{
	\maps[datatype=bibtex, overwrite]{
		\map{
			\step[fieldset=langid, fieldvalue=english]
			\step[fieldset=doi, null]
			\step[fieldset=issn, null]
			\step[fieldset=isbn, null]
			\step[fieldset=url, null]
			\step[fieldsource=language, fieldset=langid, origfieldval]
		}
	}
}

\let\cleardoublepage\clearpage

\begin{document}
	\pagestyle{empty}
		\begin{center}
			{\bfseries  Федеральное государственное автономное \\
				образовательное учреждение высшего образования\\
				<<Высшая школа экономики>>
				
			}

			\vspace{-5pt}
			\noindent\rule{\textwidth}{2pt}
			
			\vspace{50pt}
			{\Large\bfseries А.\,И.~Панов}
			
			\vspace{100pt}
			{\Huge\bfseries Кортикоморфные нейронные модели}
			
			\vspace{20pt}
			{\Large\itshape Учебно-методическое пособие}
			
			\vfill
			{\bfseries Москва\\
				Высшая школа экономики\\
				2016
			}
		\end{center}

	
	\frontmatter
	
	В пособии рассмотрены основные 
	
	\clearpage
	\tableofcontents %% содержание
		
	\mainmatter
	
	\chapter*{Введение}
	\addcontentsline{toc}{chapter}{Введение}
		

	\chapter{Нейрофизиологические данные о строении мозга}
	
	\section{Строение нейрона}

	\section{Строение коры головного мозга}
	\section{Подкорковые структуры}
	
	
	\chapter{Простые искусственные нейронные сети}
	В общем
	
	\section{Перцептрон}
	
	\section{Неокогнитрон}
	
	\section{Кресцептрон}
	
	\chapter{Иерархические искусственные нейронные сети}
		
	\section{HMAX}
	
	\section{HTM}

	\section{Сеть адаптивного резонанса}
	
	\chapter{Спайковые модели}
	
	\section{Организация памяти на спайках}
	
	
	\chapter{Глубокие модели}
	В общем о знаниях
	
	\section{Глубина сети и задачи}
	\section{Глубокие нейронные сети}
	\section{Глубокие байесовские сети}

	
	
	\chapter*{Заключение}
	\addcontentsline{toc}{chapter}{Заключение}
	Немного о целях
	\printbibliography
\end{document}