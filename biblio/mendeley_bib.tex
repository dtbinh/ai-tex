\documentclass[a4paper,12pt]{article}

\usepackage{cmap}                      % Поддержка поиска русских слов в PDF (pdflatex)
\usepackage[T2A]{fontenc}			   % Поддержка русских букв
\usepackage[utf8]{inputenc}            % Выбор языка и кодировки
\usepackage[english, russian]{babel}
\usepackage{geometry}
\usepackage{csquotes}
\usepackage{epstopdf}

\geometry{a4paper,top=2cm,bottom=2cm,left=2.5cm,right=1cm}	% Геомтерия страницы

\usepackage[
	language=auto,
	autolang=other,
	defernumbers=true,
	backend=biber,
	style=gost-numeric,
	sorting=ynt
]{biblatex}
\addbibresource{library.bib}

\DeclareSourcemap{
	\maps[datatype=bibtex, overwrite]{
		\map{
			\step[fieldset=langid, fieldvalue=english]
			\step[fieldset=doi, null]
			\step[fieldset=issn, null]
			\step[fieldset=isbn, null]
			\step[fieldset=url, null]
			\step[fieldsource=language, fieldset=langid, origfieldval]
			\step[fieldsource=mendeley-tags, fieldset=keywords, origfieldval]
		}
	}
}

\begin{document}
	\hyphenpenalty=10000
	\sloppy
	\renewcommand{\refname}{}
	
	\nocite{*}
%	\printbibliography
	\printbibliography[title={My journal articles}, keyword={mypub}]
	\printbibliography[title={My conference articles}, keyword={myconf}]
	\printbibliography[title={My thesises}, keyword={myth}]
	\printbibliography[title={My translated journal articles}, keyword={mypubtr}, resetnumbers=true]
	\printbibliography[title={Osipov's publications}, keyword={osipov}, resetnumbers=true]
	\printbibliography[title={Semiotics}, keyword={semiotics}, resetnumbers=true]
	
	\printbibliography[title={Other publications}, 
		notkeyword={mypub}, 
		notkeyword={myconf}, 
		notkeyword={myth}, 
		notkeyword={mypubtr}, 
		notkeyword={osipov}, 
		notkeyword={semiotics}, 
		resetnumbers=true]
	\par
\end{document} 	