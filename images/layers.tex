\documentclass[tikz]{standalone}

\usepackage[utf8]{inputenc} % utf8 encoding
\usepackage[english, russian]{babel}
\usepackage[T1]{fontenc} % use T1 fonts

\usepackage{tikz}
\usetikzlibrary{positioning}

% TikZ styles for drawing

\begin{document}

	\begin{tikzpicture}[
		scale=.8,
		every node/.style={minimum size=1cm},
		on grid]
		%
		\begin{scope}[
			yshift=-83,
			every node/.append style={
				yslant=0.5,xslant=-1},
			yslant=0.5,xslant=-1]
		
			\draw[step=4mm, black] (0,0) grid (5.2,5.2);
			\draw[black,thick] (0,0) rectangle (5.2,5.2);%borders
			
			\fill[black] (2.05,2.05) rectangle (2.35,2.35); % center pixel
			\fill[black] (1.65,2.05) rectangle (1.95,2.35); %left
			\fill[black] (2.45,2.05) rectangle (2.75,2.35); %right
			\fill[black] (2.05,2.45) rectangle (2.35,2.75); %top
			\fill[black] (2.05,1.95) rectangle (2.35,1.65); %bottom
			% 8 -pixel setting
			\fill[black] (1.65,2.45) rectangle (1.95,2.75); %top-left
			\fill[black] (2.45,2.45) rectangle (2.75,2.75); %top-right
			\fill[black] (2.75,1.95) rectangle (2.45,1.65); %bottom-right
			\fill[black] (1.65,1.95) rectangle (1.95,1.65); %bottom-left
			% 2. ring
			\fill[black] (1.25,1.55) rectangle (1.55,1.25); %bottom-left
			\fill[black] (0.85,1.55) rectangle (1.15,1.25); %bottom-left
			\fill[black] (0.85,1.15) rectangle (1.15,0.85); %bottom-left
			\fill[black] (1.25,0.75) rectangle (1.55,0.45); %bottom-left
		\end{scope}
		%
		\begin{scope}[
			yshift=0,
			every node/.append style={
				yslant=0.5,xslant=-1},
			yslant=0.5,xslant=-1]
			
			\fill[white,fill opacity=0.9] (0,0) rectangle (5.2,5.2);
			\draw[step=4mm, black] (0,0) grid (5.2,5.2); %grid definition
			\draw[black,thick] (0,0) rectangle (5.2,5.2);%borders
			
			\fill[black] (2.05,2.05) rectangle (2.35,2.35); % center pixel
			\fill[black] (1.65,2.05) rectangle (1.95,2.35); %left
			\fill[black] (2.45,2.05) rectangle (2.75,2.35); % right
			\fill[black] (2.05,2.45) rectangle (2.35,2.75); % top
			\fill[black] (2.05,1.95) rectangle (2.35,1.65); % bottom
			% 4 -pixel setting
			\fill[black] (1.65,2.45) rectangle (1.95,2.75); %top-left
			\fill[black] (2.45,2.45) rectangle (2.75,2.75); %top-right
			\fill[black] (2.75,1.95) rectangle (2.45,1.65); %bottom-right
			\fill[black] (1.65,1.95) rectangle (1.95,1.65); %bottom-left
			% 2. ring
			\fill[orange] (1.25,1.55) rectangle (1.55,1.25);
			\fill[orange] (0.85,1.55) rectangle (1.15,1.25);
			\fill[orange] (0.85,1.15) rectangle (1.15,0.85);
			\fill[blue] (1.25,0.75) rectangle (1.55,0.45);
		\end{scope}
		%
		% draw annotations
		%
		\draw[-latex,thick,orange](-3,5)node[left]{ }
		to[out=0,in=90] (-.4,1.4);
		\draw[-latex,thick,blue](-3,5)node[left]{ }
		to[out=0,in=90] (0.8,1.15);
		\draw[-latex,thick,black](-3,5)node[left]{3 patches}
		to[out=0,in=90] (0,2.8);
		%
		\draw[-latex,thick,black](-3,-2)node[left]{1 patch}
		to[out=0,in=200] (-1,-.9);
		\draw[thick,gray!70!black](6,4) node {4 neighbourhood rule};
		\draw[thick,gray!70!black](6,-2) node {8 neighbourhood rule};
		%
	\end{tikzpicture}

	\def\layersep{2.5cm}

	\begin{tikzpicture}[shorten >=1pt,->,draw=black!50, node distance=\layersep]
		\tikzstyle{every pin edge}=[<-,shorten <=1pt]
		\tikzstyle{neuron}=[circle,fill=black!25,minimum size=17pt,inner sep=0pt]
		\tikzstyle{input neuron}=[neuron, fill=green!50];
		\tikzstyle{output neuron}=[neuron, fill=red!50];
		\tikzstyle{hidden neuron}=[neuron, fill=blue!50];
		\tikzstyle{annot} = [text width=4em, text centered]
		
		% Draw the input layer nodes
		\foreach \name / \y in {1,...,4}
			% This is the same as writing \foreach \name / \y in {1/1,2/2,3/3,4/4}
			\node[input neuron, pin=left:Input \#\y] (I-\name) at (0,-\y) {};
		
		% Draw the hidden layer nodes
		\foreach \name / \y in {1,...,5}
			\path[yshift=0.5cm]
				node[hidden neuron] (H-\name) at (\layersep,-\y cm) {};
		
		% Draw the output layer node
		\node[output neuron,pin={[pin edge={->}]right:Output}, right of=H-3] (O) {};
		
		% Connect every node in the input layer with every node in the
		% hidden layer.
		\foreach \source in {1,...,4}
			\foreach \dest in {1,...,5}
				\path (I-\source) edge (H-\dest);
		
		% Connect every node in the hidden layer with the output layer
		\foreach \source in {1,...,5}
			\path (H-\source) edge (O);
		
		% Annotate the layers
		\node[annot,above of=H-1, node distance=1cm] (hl) {Hidden layer};
		\node[annot,left of=hl] {Input layer};
		\node[annot,right of=hl] {Output layer};
	\end{tikzpicture}
		
	\begin{tikzpicture}[scale=.9,every node/.style={minimum size=1cm},on grid]

		\begin{scope}[
			yshift=-170,
			every node/.append style={
				yslant=0.5,xslant=-1},
			yslant=0.5,xslant=-1]
			%marking border
			\draw[black,very thick] (0,0) rectangle (5,5);
			
			%drawing corners (P1,P2, P3): only 3 points needed to define a plane.
			\draw [fill=lime](0,0) circle (.1) ;
			\draw [fill=lime](0,5) circle (.1);
			\draw [fill=lime](5,0) circle (.1);
			\draw [fill=lime](5,5) circle (.1);
			
			%drawing bathymetric hypotetic countours on the bottom grid:    	
			\draw [ultra thick](0,1) parabola bend (2,2) (5,1)  ;
			\draw [dashed] (0,1.5) parabola bend (2.5,2.5) (5,1.5) ;
			\draw [dashed] (0,2) parabola bend (2.7,2.7) (5,2)  ;
			\draw [dashed] (0,2.5) parabola bend (3.5,3.5) (5,2.5)  ;
			\draw [dashed] (0,3.5)  parabola bend (2.75,4.5) (5,3.5);
			\draw [dashed] (0,4)  parabola bend (2.75,4.8) (5,4);
			\draw [dashed] (0,3)  parabola bend (2.75,3.8) (5,3);
			\draw[-latex,thick](2.8,1)node[right]{$\mathsf{Shoreline}$}
			to[out=180,in=270] (2,1.99);
		\end{scope} %end of drawing grids
				
		%slanting: production of a set of n 'laminae' to be piled up. N=number of grids.
		\begin{scope}[
			yshift=-83,
			every node/.append style={
				yslant=0.5,xslant=-1},
			yslant=0.5,xslant=-1]
		
			% opacity to prevent graphical interference
			\fill[white,fill opacity=0.9] (0,0) rectangle (5,5);
			\draw[step=4mm, black] (0,0) grid (5,5); %defining grids
			\draw[step=1mm, red!50,thin] (3,1) grid (4,2);  %Nested Grid
			\draw[black,very thick] (0,0) rectangle (5,5);%marking borders
			\fill[red] (0.05,0.05) rectangle (0.35,0.35);
			%Idem as above, for the n-th grid:
		\end{scope}
		
		\begin{scope}[
			yshift=0,
			every node/.append style={
				yslant=0.5,xslant=-1},
			yslant=0.5,xslant=-1]
			
			\fill[white,fill opacity=.9] (0,0) rectangle (5,5);
			\draw[black,very thick] (0,0) rectangle (5,5);
			\draw[step=5mm, black] (0,0) grid (5,5);
		\end{scope}
		
		\begin{scope}[
			yshift=90,
			every node/.append style={
				yslant=0.5,xslant=-1},
			yslant=0.5,xslant=-1]
			
			\fill[white,fill opacity=.9] (0,0) rectangle (5,5);
			\draw[step=10mm, black] (1,1) grid (4,4);
			\draw[black,very thick] (1,1) rectangle (4,4);
			\draw[black,dashed] (0,0) rectangle (5,5);
		\end{scope}
		
		\begin{scope}[
			yshift=170,
			every node/.append style={
				yslant=0.5,xslant=-1},
			yslant=0.5,xslant=-1]
			
			\fill[white,fill opacity=0.6] (0,0) rectangle (5,5);
			\draw[step=10mm, black] (2,2) grid (5,5);
			\draw[step=2mm, green] (2,2) grid (3,3);
			\draw[black,very thick] (2,2) rectangle (5,5);
			\draw[black,dashed] (0,0) rectangle (5,5);
		\end{scope}
		
		%putting arrows and labels:
		\draw[-latex,thick] (6.2,2) node[right]{$\mathsf{Bathymetry}$}
		to[out=180,in=90] (4,2);
		
		\draw[-latex,thick](5.8,-.3)node[right]{$\mathsf{Comp.\ G.}$}
		to[out=180,in=90] (3.9,-1);
		
		\draw[-latex,thick](5.9,5)node[right]{$\mathsf{Wind\ G.}$}
		to[out=180,in=90] (3.6,5);
		
		\draw[-latex,thick](5.9,8.4)node[right]{$\mathsf{Friction\ G.}$}
		to[out=180,in=90] (3.2,8);
		
		\draw[-latex,thick,red](5.3,-4.2)node[right]{$\mathsf{G. Cell}$}
		to[out=180,in=90] (0,-2.5);
		
		\draw[-latex,thick,red](4.3,-1.9)node[right]{$\mathsf{Nested\ G.}$}
		to[out=180,in=90] (2,-.5);
		
		\draw[-latex,thick](4,-6)node[right]{$\mathsf{Batymetry}$}
		to[out=180,in=90] (2,-5);	
		%drawing points on grid's conrners.
		\fill[black,font=\footnotesize]
		(-5,-4.3) node [above] {$P_{1}$}
		(-.3,-5.6) node [below] {$P_{2}$}
		(5.5,-4) node [above] {$P_{3}$};	
	\end{tikzpicture}
\end{document}