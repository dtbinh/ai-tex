\documentclass[tikz,convert={density=300,outext=.jpg},border=5pt]{standalone}
%\documentclass[tikz, border=5pt]{standalone}

\usepackage[utf8]{inputenc} % utf8 encoding
\usepackage[english,russian]{babel}
\usepackage[T1]{fontenc} % use T1 fonts
\usepackage{amsmath} % nice math symbols
     

\usepackage{tikz}
\usetikzlibrary{shapes,arrows,calc}

\begin{document}
	
	\begin{tikzpicture}[>=triangle 60]
		\tikzstyle{level_rect} = [draw, rectangle, rounded corners=5ex, minimum height = 60mm, minimum width = 150mm,fill=black!3];
		\tikzstyle{level_model} =[draw, ellipse, minimum width = 80mm, minimum height = 15mm,align = center];
		\tikzstyle{control_block} = [draw, rectangle, minimum height = 20mm, minimum width = 25mm, align=center]
		\tikzstyle{traekt_block} = [draw, rectangle, minimum height = 15mm, minimum width = 25mm, align=center]
		\tikzstyle{cons_block} = [draw, rectangle, minimum height = 13mm, minimum width = 30mm, align=center]
		\tikzstyle{misc_block} = [draw, rectangle, minimum height = 10mm, minimum width = 50mm, align=center]
		
		\node[misc_block] (messages) at (8, 20.5) {Другие участники\\коалиции};
		\node[misc_block] (task) at (-8, 20.5) {Формирование общей\\коалиционной задачи};
		

		\node[level_rect] (level_1) at (0, 16) {};
		\node[font=\huge] at(0, 18.5) {Стратегический уровень};		
		\node[level_model, minimum height = 10mm] (model_1) at (0, 17.5) {Знаковая картина мира};
		\node[cons_block] (sit_decr) at (-4, 15.5) {Построение\\описания\\текущей ситуации};
		\node[cons_block] (beh_plan) at (4, 15.5) {Планирование\\поведение};
		\node[cons_block] (replaner) at (0, 14) {Процедура\\перепланирования};
		
		\draw[->,rounded corners] ([xshift=-3mm]model_1.south) |- ([yshift=5mm]sit_decr.east);
		\draw[->,rounded corners] ([xshift=3mm]model_1.south) |- ([yshift=5mm]beh_plan.west);
		\draw[->] ([yshift=-5mm]beh_plan.west) node[align = center, xshift = -25mm, above] {планируемая ситуация}-- ([yshift=-5mm]sit_decr.east);
		
		\draw[<->,rounded corners] (messages.south) node[align = center, yshift = -20mm, right] {обмен\\сообщениями\\по протоколу}|- (beh_plan.east);
		\draw[->,rounded corners] (task.south) node[align = center, yshift = -15mm, left] {задача\\деятельности}|- (model_1.west);
		\draw[->, rounded corners] (replaner.east) node[align = center, xshift = 13mm, below] {обновленный\\список знаков}-| (beh_plan.south);
		\draw[->, rounded corners] (sit_decr.south) |- node[align = center, xshift = 12mm, below] {текущая\\ситуация} (replaner.west);
		
		

		\node[level_rect] (level_2) at (0, 8) {};
		\node[font=\huge] at(0, 10.5) {Тактический уровень};		
		\node[traekt_block] (slam) at (-5, 9) {Картирование и\\локализация};		
		\node[traekt_block] (pred_tr) at (-4, 7) {Прогнозирование\\траектории};		
		\node[traekt_block] (build_tr) at (5, 9) {Планирование\\траектории};		
		\node[traekt_block] (monitor) at (4, 7) {Мониторинг\\выполнения плана};
		\node[level_model, minimum width = 60mm] (model_3) at (0, 9) {Пространственная\\модель местности};
		
		
		\draw[->,rounded corners] ([yshift=-3mm]sit_decr.west) -| ($(sit_decr.west)+(-13mm,-35mm)$) node[align=center, right] {целевая область\\и временные\\ограничения} |- (pred_tr.west);
		\draw[->,rounded corners] (build_tr.north) |- ($(build_tr.north)+(-25mm, 17mm)$) node[align=center, above] {траектория\\построена\\либо нет} -| (replaner.south);
		\draw[<-, rounded corners] ([xshift=3mm]pred_tr.south) |- ($(pred_tr.south)+(40mm,-5mm)$) node[align = center, above] {запрос на\\перепланирование} -| (monitor.south);
		\draw[->, rounded corners] (model_3.south) |- (monitor.west);
		\draw[<-, rounded corners] (model_3.east) -- (build_tr.west);
		\draw[<-, rounded corners] (model_3.west) -- (slam.east);


		\node[level_rect] (level_3) at (0, 0) {};
		\node[font=\huge] at(0, 2.5) {Реактивный уровень};		
		\node[control_block] (geom_3) at (-5.7, 0.5) {Расчёт\\геометрических\\ограничений};
		\node[control_block] (integr_3) at (-0.7, 0.5) {Интегрирование\\уравнений\\модели\\динамики};
		\node[control_block] (signal_3) at (5.7, 0.5) {Выработка\\управляющего\\сигнала и\\анализ\\ динамики};
		\node[level_model] (model_3) at (-0.7, -2) {Модель динамики};
				
		\draw[->] ([yshift=3mm]geom_3.east) -- node[align = center, auto] {параметры\\движения} ([yshift=3mm]integr_3.west);
		\draw[<-] ([yshift=-3mm]geom_3.east) -- node[align = center, below] {фазовые\\координаты} ([yshift=-3mm]integr_3.west);
		
		\draw[->] ([yshift=3mm]integr_3.east) -- node[align = center, auto] {фазовые\\координаты и\\параметры движения} ([yshift=3mm]signal_3.west);
		\draw[<-] ([yshift=-3mm]integr_3.east) -- node[align = center, below] {управляющий\\сигнал} ([yshift=-3mm]signal_3.west);
		
		\draw[<-] (integr_3.south) -- ([xshift=0mm]model_3.north);
		
		\draw[->,rounded corners] ([xshift=-3mm]pred_tr.south) node[align=center,yshift=-7mm,left] {параметры\\движения}|- ($(pred_tr.south)+(-7mm,-20mm)$) -| ([xshift=-5mm]geom_3.north);
		\draw[->,rounded corners] ([xshift=5mm]geom_3.north) |- node[align=center, xshift = 50mm, auto] {геометрические\\ограничения} ($(geom_3.north)+(7mm,20mm)$) -| ([xshift=10mm]build_tr.south);
		\draw[->,rounded corners] (build_tr.east) -| node[align=center,yshift=-50mm,right] {желаемая\\траектория} ([xshift=10mm]signal_3.north);
		\draw[->,rounded corners] ([xshift=-7mm]signal_3.north) -- node[align=center, yshift = 2mm, left] {превышен ли\\порог ошибки} ([xshift=10mm]monitor.south);
	
		\node[misc_block] (control_meh) at (8, -4.5) {Органы управления};
		\node[misc_block] (sensors) at (-8, -4.5) {Сенсоры};
		
		\draw[->,rounded corners] (signal_3.east) -| (control_meh.north) node[align=center, yshift = 20mm, right] {управляющий\\сигнал};
		\draw[->,rounded corners] (sensors.east) node[align = center, xshift = 50mm, above] {фазовые координаты} -| ([xshift=-7mm]signal_3.south);
		\draw[->,rounded corners] ([xshift=-13mm]sensors.north) node[yshift = 115mm, rotate=90, above] {информация об объектах и действиях} |- ([yshift=3mm]sit_decr.west);
		\draw[->,rounded corners] ([xshift=-5mm]sensors.north) node[yshift = 75mm, rotate=90, above] {свойства внешней среды} |- (slam.west);
	\end{tikzpicture}

\end{document}