\documentclass[12pt]{report}
\usepackage[top=2cm, bottom=2cm, left=3cm, right=1.5cm]{geometry}

\usepackage{fontspec}
\usepackage{polyglossia}
\setdefaultlanguage{russian}
\newcommand{\docfont}{Times New Roman}
\setmainfont[Ligatures=TeX]{\docfont}
\newfontfamily\cyrillicfont[Ligatures=TeX]{\docfont}
\newfontfamily\cyrillicfontsf[Ligatures=TeX]{\docfont}
\newfontfamily\cyrillicfonttt[Ligatures=TeX]{\docfont}

\usepackage[plain]{algorithm}
\usepackage[noend]{algpseudocode}
\renewcommand{\algorithmiccomment}[1]{{\quad\footnotesize // #1}}
\renewcommand{\algorithmicrequire}{\textbf{Input:}}
\renewcommand{\algorithmicensure}{\textbf{Output:}}

\usepackage{amsmath,amsfonts,amsthm,amssymb} % nice math symbols
\newtheorem{definition}{Определение}

\usepackage{hyperref}
\hypersetup{%
	pdfencoding=auto,
	pdfauthor={Александр Панов},
	pdftitle={Отчет РФФИ мол_а_дк}
}

\usepackage[
	autolang=hyphen,
	language=auto,
	autolang=other,
	backend=biber,
	style=gost-numeric
]{biblatex}
\addbibresource{rfbr_moladk_2017.bib}

\DeclareSourcemap{
	\maps[datatype=bibtex, overwrite]{
		\map{
			\step[fieldset=langid, fieldvalue=english]
			\step[fieldset=doi, null]
			\step[fieldset=issn, null]
			\step[fieldset=isbn, null]
			\step[fieldset=url, null]
			\step[fieldsource=language, fieldset=langid, origfieldval]
		}
	}
}

\usepackage{graphicx}
\graphicspath{{../../images/}}

\title{Развернутый научный отчет по проекту \textnumero 16-37-60055 (2-й этап, 2017 г.)}
\author{А.\,И.~Панов}

\begin{document}
	\maketitle
	
	\section*{Аннотация}
	

	\section*{Заявленные цели Проекта на 2017~г.}
	Второй год работ по проекту (2017 г.): 
	\begin{enumerate}
		\item Исследование процессов восприятия и категоризации в первичных зонах коры головного мозга человека, разработка модели приобретения знаний на основе нейрофизиологических данных.
		\item Разработка нового метода машинного обучения, построенного с использованием нейрофизиологических данных, работающего на массиве прецедентной информации о процессах, действиях и их результатах во внешней среде и извлекающего причинно-следственных связи.
		\item Подготовка публикаций в ведущих периодических изданиях, включенных в одну из систем цитирования (библиографических баз) Web of Science, Scopus, РИНЦ.
	\end{enumerate}
	
	\section*{Полученные за 2017~г. результаты с описанием методов и подходов, использованных в ходе выполнения проекта}
	За отчетный период в ходе работ по проекту была разработана модель приобретения знаний на основе нейрофизиологических данных о строении первичных отделов коры головного мозга человека и представлен новый алгоритм машинного обучения на основе прецедентов планирования и выполнения действий и подкрепления, получаемого от внешней среды, с извлечением причинно-следственной информации и формированием новых правил. В ходе работы были рассмотрены модельные задачи по планированию пути в лабиринте и планированию действий в домене <<Мир кубиков>>. В продолжение работ первого года, был развит алгоритм целеполагания и построен новый алгоритм индивидуального планирования - GoalMAP.
	
	В ходе работ была рассмотрена задача автоматического формирования правил интеллектуальным агентом в процессе обучения планированию действий. Была приведена постановка задачи при использовании варианта обучения с подкреплением. В качестве примера рассмотрена дискретная задача планирования траектории на плоскости с препятствиями, в которой формируется план достижения целевого положения из начального. Был проведен ряд экспериментов с использованием кортикоморфных алгоритмов обучения: аналога иерархической временной памяти и различных вариантов глубоких нейронных сетей.
	
	
	\section*{Участие в научных мероприятиях по тематике Проекта за 2017~г.}
	\begin{enumerate}
		\item Доклад <<Знаковые модели обучения в задаче планирования поведения>>, Лабораторная ФКН ВШЭ, Москва, 24 января 2017.
		\item Доклад <<Автоматическое формирование правил перемещения с использованием обучения с подкреплением>, Седьмая Международная конференция <<Системный анализ и информационные технологии>> (САИТ-2017), Свтелогорск, 13-18 июня 2017.
		\item Доклад <<Планирование действий коалицией агентов: коммуникационный аспект>>, Четвертый Всероссийский научно-практический семинар <<Беспилотные транспортные средства с элементами искусственного интеллекта>> (БТС-ИИ-2017), Казань, 22-23 сентября 2017.
		\item Доклад <<Теория деятельности и когнитивные архитектуры>>, Научный семинар ИИКС НИЯУ МИФИ, Москва, 26 октября.
		\item Доклад <<Методы стратегического управления робототехнической  системы в составе коалиции>>, Научный семинар <<Интеллектуальные системы управления роботов>> ФИЦ ИУ РАН, Москва, 7 декабря.
	\end{enumerate}
	
	\section*{Библиографический список всех публикаций по Проекту, опубликованных за 2017~г.}
	\nocite{*}
	\printbibliography[heading=none,keyword={mypub, myconf},resetnumbers=true]
\end{document}