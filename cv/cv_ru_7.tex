\documentclass[11pt,a4paper,sans]{moderncv}
\moderncvstyle{casual}
\moderncvcolor{blue}

\usepackage[scale=0.75]{geometry}

\usepackage{fontspec}
\usepackage{polyglossia}
\setdefaultlanguage{english}
\setmainlanguage{russian}
\setmainfont[Ligatures=TeX]{Calibri}
\newfontfamily\cyrillicfont[Ligatures=TeX]{Calibri}
\newfontfamily\cyrillicfontrm[Ligatures=TeX]{Calibri}
\newfontfamily\cyrillicfontsf[Ligatures=TeX]{Calibri}
\newfontfamily\cyrillicfonttt[Ligatures=TeX]{Calibri}
%----------------------------------------------------------------------------------------
%	NAME AND CONTACT INFORMATION SECTION
%----------------------------------------------------------------------------------------

\firstname{Александр} % Your first name
\familyname{Панов} % Your last name

% All information in this block is optional, comment out any lines you don't need
\title{Curriculum Vitae}
\address{117312, Москва}{пр-т 60-летия Октября, 9}
\mobile{+7 (916) 144 5255}
\phone{+7 (499) 137 7310}
%\fax{(000) 111 1113}
\email{pan@isa.ru,apanov@hse.ru}
\homepage{hse.ru/staff/apanov}{hse.ru/staff/apanov}
%\extrainfo{additional information}
\photo[100pt][0.4pt]{../images/mine/cv_photo_2} 
%\quote{"A witty and playful quotation" - John Smith}

\hyphenation{РФФИ}
\usepackage[
	language=auto,
	autolang=langname,
	defernumbers=true,
	backend=biber,	
	bibstyle=gost-numeric,
	sorting=ynt,
	maxbibnames=15
]{biblatex}

\addbibresource{cv_ru_7.bib}
\DeclareSourcemap{
	\maps[datatype=bibtex, overwrite]{
		\map{
			\step[fieldset=langid, fieldvalue=english]
			\step[fieldset=doi, null]
			\step[fieldset=issn, null]
			\step[fieldset=isbn, null]
			\step[fieldset=url, null]
			\step[fieldsource=language, fieldset=langid, origfieldval]
		}
	}
}

\begin{document}

\makecvtitle % Print the CV title

%----------------------------------------------------------------------------------------
%	EDUCATION SECTION
%----------------------------------------------------------------------------------------


\section{Образование}

\cventry{2011--2015}{Кандидат физико-математических наук по направлению <<05.13.17 – Теоретические основы информатики>>}{Институт системного анализа РАН}{}{Москва}{Тема диссертации <<Исследование методов, разработка моделей и алгоритмов формирования элементов знаковой картины мира субъекта деятельности>>, науч. руководитель – Г.\,С.~Осипов} 

\cventry{2009--2011}{Магистр прикладных математики и физики по направлению <<Прикладные математика и физика>>}{Московский физико-технический институт}{}{Москва}{Тема диссертации <<Исследование и моделирование поведения коллектива интеллектуальных агентов с различной функциональностью>>, науч. руководитель – Г.\,С.~Осипов}

\cventry{2005--2009}{Бакалавр физики по направлению <<Физика>>}{Новосибирский государственный университет}{Новосибирск}{}{}


\section{Опыт научно-педагогической работы}

\cventry{2015--по н.в.}{Доцент}{Высшая школа экономики}{факультет компьютерных наук}{Москва}{Семинарские занятия, майнор <<Анализ данных>>.}
\cventry{2011--по н.в.}{Доцент}{Московский физико-технический институт}{кафедра информатики и вычислительной математики}{Москва}{Семинарские занятия, <<Основы операционных систем>> и <<Основы объектно-ориентированного программирования>>.}
\cventry{2011--2016}{Ассистент}{Российский университет дружбы народов}{кафедра информационных технологий факультета естественных и физико-математических наук}{Москва}{Лекции, <<Интеллектуальные динамические системы>>, <<Теоретические основы информатики>>, <<Интеллектуальный анализ данных>>.}  % Arguments not required can be left empty

\section{Опыт научной работы}

\cventry{2015--по н.в.}{Научный сотрудник}{\textsc{Высшая школа экономики}}{Лаборатория процессно-ориентированных информационных систем}{Москва}{
	\begin{itemize}
		\item \textit{Компьютерное когнитивное моделирование}: исследование методов обучению в задаче планирования поведения на основе знаковой картины мира.
	\end{itemize}	
	}  % Arguments not required can be left empty
\cventry{2010--по н.в.}{Старший научный сотрудник}{\textsc{ФИЦ <<Информатика и управление>> РАН}}{лаборатория <<Динамические интеллектуальные системы>>}{Moscow, Russia}{
	\begin{itemize}
		\item \textit{Компьютерное когнитивное моделирование}: исследование и моделирование процессов восприятия, планирования поведения, целеполагания и других высших когнитивных функций человека.
		\begin{itemize}
			\item Предложены модели некоторых когнитивных функций на основе знакового опосредования.
			\item Исследован процесс образования элементов картины мира субъекта деятельности (знаков).
			\item Предложены и исследованы модели компонент знака на основе нейрофизиологических данных.
		\end{itemize}
		\item \textit{Машинное обучение и распознавание изображений}: разработка алгоритмов логического и гибридного методов анализа данных, разработка биологически правдоподобных алгоритмов распознавания изображений и сцен.
		\begin{itemize}
			\item Разработан гибридный метод выявления причинно-следственных связей в массиве слабоструктурированной информации. 
		\end{itemize}
		\item \textit{Многоагентные системы и системы управления}: исследование распределения ролей в коллективе агентов, разработка многоуровневых архитектур управления коллективами сложных технических объектов.
		\begin{itemize}
			\item Разработана многоуровневая система управления коллективом БПЛА STRL. 
		\end{itemize}
	\end{itemize}
	}


%----------------------------------------------------------------------------------------
%	GRANTS
%----------------------------------------------------------------------------------------

\section{Научные гранты}

\subsection{В качестве руководителя}

	\cventry{2016--по н.в.}{Гранты для постдоков}{Российский фонд фундаментальных исследований (РФФИ)}{}{}{Исследование механизмов и построение моделей обучения, основанных на знаковых представлениях, в задаче планирования коллективного поведения.}
	
	\cventry{2016--по н.в.}{Гранты ориентированных фундаментальных исследований}{Российский фонд фундаментальных исследований (РФФИ)}{}{}{Разработка новых методов формирования баз знаний, поиска и адаптации прецедентов о существующих научно-технических решениях и технологиях по их текстовым описаниям на основе теории семантических сетей.}
	
	%------------------------------------------------
	
	\cventry{2014--2015}{Гранты молодым ученым}{Российский фонд фундаментальных исследований (РФФИ)}{}{}{Исследование механизмов распределения ролей в коллективе интеллектуальных агентов при решении задачи выявления причинно-следственных связей на множестве событий предметной области.}
	
	%------------------------------------------------
	
\subsection{В качестве ответственного исполнителя}
	
	\cventry{2016--по н.в.}{Гранты по приоритетным направлениям исследований}{Российский научный фонд (РНФ)}{руководитель: Г.\,С.~Осипов}{}{Создание теории, методов и моделей децентрализованного управления поведением коллективов когнитивных робототехнических систем в недетерминированной среде.}
	
	\cventry{2015--по н.в.}{Инициативные проекты}{Российский фонд фундаментальных исследований (\mbox{РФФИ})}{руководитель: Г.\,С.~Осипов}{}{Нейрофизиологические и психологические основания знаковой картины мира и моделей когнитивных функций.}
	
	\cventry{2012--2014}{Инициативные проекты}{Российский фонд фундаментальных исследований (\mbox{РФФИ})}{руководитель: Г.\,С.~Осипов}{}{Исследование управляемой сознанием деятельности и моделирование поведения и ролевой структуры коллектива интеллектуальных агентов.}	

%----------------------------------------------------------------------------------------
%	INTERESTS SECTION
%----------------------------------------------------------------------------------------

\section{Научные интересы}

\renewcommand{\listitemsymbol}{-~} % Changes the symbol used for lists

\cvlistdoubleitem{компьютерное когнитивное моделирование}{многоагентные системы}
\cvlistdoubleitem{семиотика}{моделирование внимания}
\cvlistdoubleitem{распознавание образов}{машинное обучение}


%\section{Grants and Awards}
%
%\cvitem{Title}{\emph{Money Is The Root Of All Evil -- Or Is It?}}
%\cvitem{Supervisors}{Professor James Smith \& Associate Professor Jane Smith}
%\cvitem{Description}{This thesis explored the idea that money has been the cause of untold anguish and suffering in the world. I found that it has, in fact, not.}

\section{Научные сообщества и редколлегии}

\cvitem{2016--по н.в.}{Член редколлегии журнала \textit{Biologically Inspired Cognitive Architectures}: \href{http://www.journals.elsevier.com/biologically-inspired-cognitive-architectures/}{BICA Journal}.}
\cvitem{2016--по н.в.}{Член Сообщества биологически инспирированных когнитивных архитектур: \href{http://bicasociety.org}{BICA Society}.}
\cvitem{2016}{Ответственный секретарь организацинного коммитета Первой международной школы по биологически инспирированным когнитивным архитектурам: \href{school.bicasociety.org}{Fierces on BICA 2016}.}
\cvitem{2016}{Сопредседатель организацинного коммитета Седьмой Международной конференции по биологически инспирированным когнитивным архитектурам: \href{bica2016.bicasociety.org}{BICA 2016}.}
\cvitem{2015--по н.в.}{Член Российской ассоциации искусственного интеллекта: \href{www.raai.org}{РААИ}.}
\cvitem{2015--по н.в.}{Член рабочей группы Нейронет Национальной технологической инициативы: \href{www.asi.ru/nti/}{НТИ}.}

%----------------------------------------------------------------------------------------
%	COMPUTER SKILLS SECTION
%----------------------------------------------------------------------------------------

%\section{Computer skills}
%
%\cvitem{Basic}{\textsc{java}, Adobe Illustrator}
%\cvitem{Intermediate}{\textsc{python}, \textsc{html}, \LaTeX, OpenOffice, Linux, Microsoft Windows}
%\cvitem{Advanced}{Computer Hardware and Support}

%----------------------------------------------------------------------------------------
%	LANGUAGES SECTION
%----------------------------------------------------------------------------------------

%\section{Languages}
%
%\cvitemwithcomment{English}{Mothertongue}{}
%\cvitemwithcomment{Spanish}{Intermediate}{Conversationally fluent}
%\cvitemwithcomment{Dutch}{Basic}{Basic words and phrases only}

\nocite{*}
\printbibliography[title={Основные публикации}]

\end{document}