\documentclass[11pt,a4paper,sans]{moderncv} % Font sizes: 10, 11, or 12; paper sizes: a4paper, letterpaper, a5paper, legalpaper, executivepaper or landscape; font families: sans or roman

\moderncvstyle{casual} % CV theme - options include: 'casual' (default), 'classic', 'oldstyle' and 'banking'
\moderncvcolor{blue} % CV color - options include: 'blue' (default), 'orange', 'green', 'red', 'purple', 'grey' and 'black'

\usepackage[scale=0.75]{geometry} % Reduce document margins
%\setlength{\hintscolumnwidth}{3cm} % Uncomment to change the width of the dates column
%\setlength{\makecvtitlenamewidth}{10cm} % For the 'classic' style, uncomment to adjust the width of the space allocated to your name

%----------------------------------------------------------------------------------------
%	NAME AND CONTACT INFORMATION SECTION
%----------------------------------------------------------------------------------------

\firstname{Aleksandr I.} % Your first name
\familyname{Panov} % Your last name

% All information in this block is optional, comment out any lines you don't need
\title{Curriculum Vitae}
\address{pr-t 60-letiya Octyabrya, 9}{Moscow, Russia}
\mobile{+7 (916) 144 5255}
\phone{+7 (499) 137 7310}
%\fax{(000) 111 1113}
\email{pan@isa.ru}
\homepage{hse.ru/en/staff/apanov}{hse.ru/en/staff/apanov} % The first argument is the url for the clickable link, the second argument is the url displayed in the template - this allows special characters to be displayed such as the tilde in this example
%\extrainfo{additional information}
\photo[70pt][0.4pt]{../images/mine/cv_photo} % The first bracket is the picture height, the second is the thickness of the frame around the picture (0pt for no frame)
%\quote{"A witty and playful quotation" - John Smith}

%----------------------------------------------------------------------------------------
\usepackage[
	language=auto,
	autolang=other,
	defernumbers=true,
	backend=biber,
	sorting=ynt,
	maxbibnames=15
]{biblatex}

\addbibresource{cv_7.bib}

\DeclareSourcemap{
	\maps[datatype=bibtex, overwrite]{
		\map{
			\step[fieldset=doi, null]
			\step[fieldset=issn, null]
			\step[fieldset=isbn, null]
			\step[fieldset=url, null]
			\step[fieldsource=mendeley-tags, fieldset=keywords, origfieldval]
		}
	}
}

\begin{document}

\makecvtitle % Print the CV title

%----------------------------------------------------------------------------------------
%	EDUCATION SECTION
%----------------------------------------------------------------------------------------


\section{Educational Background}

\cventry{2011--2015}{Ph.D. in Theoretical Bases of Computer Science}{Institute for Systems Analysis}{}{Moscow, Russia}{Specialized in modelling of goal-oriented behavior of intelligent agents and their coalitions}  % Arguments not required can be left empty
\cventry{2009--2011}{Master of Applied Mathematics and Physics}{Moscow Institute of Physics and Technology}{}{Moscow, Russia}{Specialized in logical methods (AQ, JSM) of data mining and multiagent systems}  % Arguments not required can be left empty
\cventry{2005--2009}{Bachelor of Physics}{Novosibirsk State University}{Novosibirsk, Russia}{}{Specialized in semantic integration of databases}


\section{Research Experience}

\cventry{2015--Present}{Research Fellow}{\textsc{National Research University Higher School of Economics}}{Laboratory of Process-Aware Information Systems (PAIS Lab)}{Moscow, Russia}{
	\begin{itemize}
		\item Investigation of learning mechanisms based on sign representations in the problem of collective behavior planning.
	\end{itemize}	
	}  % Arguments not required can be left empty
\cventry{2010--Present}{Senior Research Fellow}{\textsc{Federal Research Center ``Computer Science and Control'' of Russian Academy of Sciences}}{Laboratory of Dynamic Intelligent Systems}{Moscow, Russia}{
	\begin{itemize}
		\item Cognitive modelling:
		\begin{itemize}
			\item Proposed the models of a number of cognitive functions of consciousness based on the so-called ``semiotic mediation''.
			\item Proposed a model of the sign and investigated procedures of the sign formation.
			\item Proposed biologically inspired models of sign components: image, significance and personal meaning.
		\end{itemize}
		\item Maching learning and multi-agent systems:
		\begin{itemize}
			\item Developed the composite logical method to extract cause-effect relationships.
			\item Investigated some models of coalition formation and role distribution in the collective of intelligent agents.
		\end{itemize}
	\end{itemize}
	}

\section{Teaching Experience}

\cventry{2015--Present}{Associate Professor}{National Research University Higher School of Economics}{Faculty of Computer Science}{Moscow, Russia}{Seminar on Intelligent Data Mining}
\cventry{2011--Present}{Associate Professor}{Moscow Institute of Physics and Technology}{Department of Computer Science}{Moscow, Russia}{Seminar on Basis of Operation Systems and Basis of Object-Oriented Programming}
\cventry{2011--2016}{Assistant Lecturer}{Peoples' Friendship University of Russia}{Department of Computer Science}{Moscow, Russia}{Lectures on Intelligent Dynamic Systems, Theoretical Computer Science and Intelligent Data Analysis}  % Arguments not required can be left empty

%----------------------------------------------------------------------------------------
%	GRANTS
%----------------------------------------------------------------------------------------

\section{Research Grants}

\subsection{As a head}

	\cventry{2016--Present}{Grant for postdocs}{Russian Foundation for Basic Research (RFBR)}{}{}{Investigation of learning mechanisms based on sign representations in the problem of collective behavior planning.}
	
	\cventry{2016--Present}{Oriented basic research}{Russian Foundation for Basic Research (RFBR)}{}{}{Development of new methods for knowledge base construction, search and adaptation of cases for scientific-technical solutions and technologies using their text descriptions based on semantic networks.}
	
	%------------------------------------------------
	
	\cventry{2014--2015}{Grant for young scientists}{Russian Foundation for Basic Research (RFBR)}{}{}{Investigate of mechanisms for the distribution of roles in the collective of intelligent agents to solve the problem to identify cause-and-effect relationships on the set of domain events.}
	
	%------------------------------------------------
	
\subsection{As a senior researcher}
	
	\cventry{2016--Present}{Grant in priority thematic research areas}{Russian Science Foundation (RSF)}{research adviser: Prof. Gennady S. Osipov}{}{Creation of theory, methods and models for distributed control of behavior of cognitive robotic systems and their coalitions in nondeterministic environment.}
	
	\cventry{2015--Present}{Individual grant}{Russian Foundation for Basic Research (RFBR)}{research adviser: Prof. Gennady S. Osipov}{}{Neurophysiological and psychological foundations of sign models of the world and cognitive functions.}
	
	\cventry{2015--Present}{Grant for young headers}{Russian Foundation for Basic Research (RFBR)}{research adviser: Ph.D. Konstantin S. Yakovlev}{}{Path planning methods and algorithms in the context of cooperative task solving for intelligent agents.}	

%----------------------------------------------------------------------------------------
%	INTERESTS SECTION
%----------------------------------------------------------------------------------------

\section{Research Interests}

\renewcommand{\listitemsymbol}{-~} % Changes the symbol used for lists

\cvlistdoubleitem{Modelling of cognitive processes}{Multi-agent systems}
\cvlistdoubleitem{Semiotics}{Modelling of attention}
\cvlistdoubleitem{Pattern recognition}{Machine learning}


%\section{Grants and Awards}
%
%\cvitem{Title}{\emph{Money Is The Root Of All Evil -- Or Is It?}}
%\cvitem{Supervisors}{Professor James Smith \& Associate Professor Jane Smith}
%\cvitem{Description}{This thesis explored the idea that money has been the cause of untold anguish and suffering in the world. I found that it has, in fact, not.}

\section{Committees and Councils}

\cvitem{2016--present}{Member of the Editorial Board of the \textit{Biologically Inspired Cognitive Architectures}: BICA Journal, \url{http://www.journals.elsevier.com/biologically-inspired-cognitive-architectures/}}
\cvitem{2016--Present}{Member of The Biologically Inspired Cognitive Architectures Society: BICA Society, \url{bicasociety.org}}
\cvitem{2016--Present}{Executive Chair of the Organizing Committee of the First International Early Research Career Enhancement School on Biologically Inspired Cognitive Architectures: Fierces on BICA, \url{school.bicasociety.org}}
\cvitem{2015--Present}{Regular Fellow of the Russian Association of the Artificial Intelligence: RAAI, \url{www.raai.org}}
\cvitem{2015--Present}{Member of the NEURONET workgroup of the National Technology Initiative: NTI, \url{www.asi.ru/nti/}}

%----------------------------------------------------------------------------------------
%	COMPUTER SKILLS SECTION
%----------------------------------------------------------------------------------------

%\section{Computer skills}
%
%\cvitem{Basic}{\textsc{java}, Adobe Illustrator}
%\cvitem{Intermediate}{\textsc{python}, \textsc{html}, \LaTeX, OpenOffice, Linux, Microsoft Windows}
%\cvitem{Advanced}{Computer Hardware and Support}

%----------------------------------------------------------------------------------------
%	LANGUAGES SECTION
%----------------------------------------------------------------------------------------

%\section{Languages}
%
%\cvitemwithcomment{English}{Mothertongue}{}
%\cvitemwithcomment{Spanish}{Intermediate}{Conversationally fluent}
%\cvitemwithcomment{Dutch}{Basic}{Basic words and phrases only}

\nocite{*}
\printbibliography[title={Selected Publications}]

\end{document}