%%% Макет страницы %%%
\geometry{a4paper,top=2cm,bottom=2cm,left=2.5cm,right=1cm}

\fancyhf{}
\renewcommand{\headrulewidth}{0pt}
\fancyhead[C]{\fontsize{14}{\baselineskip} \selectfont  \thepage}
\pagestyle{fancy}
% redefine the plain pagestyle
\fancypagestyle{plain}{%
	\fancyhf{} % clear all header and footer fields
	\fancyhead[C]{\fontsize{14}{\baselineskip} \selectfont \thepage} % except the center
}

%%% Кодировки и шрифты %%%
\renewcommand{\rmdefault}{ftm} % Включаем Times New Roman
\renewcommand{\baselinestretch}{1.5}

%%% Выравнивание и переносы %%%
\sloppy					% Избавляемся от переполнений
\clubpenalty=10000		% Запрещаем разрыв страницы после первой строки абзаца
\widowpenalty=10000		% Запрещаем разрыв страницы после последней строки абзаца
\hyphenpenalty=10000		% Запрещаем переносы во всем тексте

%%% Библиография %%%
\makeatletter
%\bibliographystyle{utf8gost705u}	% Оформляем библиографию в соответствии с ГОСТ 7.0.5
\bibliographystyle{gost2008p}
\renewcommand{\@biblabel}[1]{#1.}	% Заменяем библиографию с квадратных скобок на точку:
\makeatother

%%% Изображения %%%
\graphicspath{{../images/}} % Пути к изображениям

%%% Цвета гиперссылок %%%
\definecolor{linkcolor}{rgb}{0.9,0,0}
\definecolor{citecolor}{rgb}{0,0.6,0}
\definecolor{urlcolor}{rgb}{0,0,1}
\hypersetup{
    colorlinks, linkcolor={linkcolor},
    citecolor={citecolor}, urlcolor={urlcolor}
}

%%% Оглавление %%%
\renewcommand{\cftchapdotsep}{\cftdotsep}

%%% Теормоподобные окружения %%%
\theoremstyle{plain}
\newtheorem{Theorem}{Теорема}
\newtheorem{Lemma}[Theorem]{Лемма}
\newtheorem{Pred}{Утверждение}
\newtheorem{Corollary}{Следствие}
\newtheorem{Def}{Определение}
\newtheorem{Axiom}{Аксиома}
\newtheorem{Hypothesis}{Гипотеза}
\newtheorem{Assumption}{Предположение}
\newtheorem{Problem}{Задача}
\newtheorem{Example}{Пример}
\newtheorem{Fact}{Факт}
\newtheorem{Remark}{Замечание}
\newtheorem{Rule}{Правило}
\newtheorem{Condition}{Условие}
\newenvironment{Proof}%
    {\par\noindent{\bf Доказательство.}}%
    {\hfill$\scriptstyle\blacksquare$}

% Нумерованный перечень со скобками
\renewcommand\labelenumi{\theenumi )}

% чтобы поставить точечку после номера алгоритма в \caption:
\captionsetup[ruled]{labelsep=period}

% настройка заголовков
\titleformat{\chapter}[display]
{\normalfont\huge\bfseries\centering}
{\chaptertitlename\ \thechapter}{0pt}{\Huge}

\titleformat{\section}[block]
	{\normalfont\LARGE\bfseries\centering}
	{\thesection.}{10pt}{\LARGE}

\titleformat{\subsection}[block]
	{\normalfont\large\bfseries\centering}
	{\thesubsection.}{10pt}{\large}

\titlespacing*{\chapter}{0pt}{-10pt}{1\baselineskip}
\titlespacing*{\section}{0pt}{1\baselineskip}{1\baselineskip}
\titlespacing*{\subsection}{0pt}{1\baselineskip}{1\baselineskip}
\parindent=1.2cm		% абзацный отступ

% центрируем заголовок оглавления
\renewcommand{\cfttoctitlefont}{\hfill\Huge\bfseries}
\renewcommand{\cftaftertoctitle}{\hfill}

\DeclareMathOperator*{\argmax}{arg\,max}
\algnewcommand\And{\textbf{and}}

% Счетчики
\newcounter{citnum}
\def\oldbibitem{} \let\oldbibitem=\bibitem
\def\bibitem{\stepcounter{citnum}\oldbibitem}

\renewcommand{\thesubfigure}{\Roman{subfigure}} % именование подкартинок