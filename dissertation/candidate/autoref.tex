\documentclass[%
autoref,        % тип документа
natbib,         % использовать пакет natbib для "сжатия" цитирований
href,           % использовать пакет hyperref для создания гиперссылок
facsimile,      % отображать факсимиле диссертанта и ученого секретаря
colorlinks=true % цветные гиперссылки
%,fixint=false  % отключить прямые знаки интегралов
%,times         % шрифт Times как основной
%,classified    % гриф секретности
]{disser}

\usepackage[
  a4paper, mag=1000,
  left=2.5cm, right=1cm, top=2cm, bottom=2cm, headsep=0.7cm, footskip=1cm
]{geometry}
\usepackage[T2A]{fontenc}
\usepackage[utf8]{inputenc}
\usepackage[english,russian]{babel}
\usepackage{tabularx}
\ifpdf\usepackage{epstopdf}\fi

% Номера страниц снизу и по центру
%\pagestyle{footcenter}
%\chapterpagestyle{footcenter}

% Точка с запятой в качестве разделителя между номерами цитирований
%\setcitestyle{semicolon}

% Поддержка нескольких списков литературы в одном документе
\usepackage{multibib}
% Создание команд для цитирования собственных работ диссертанта
% в отдельном разделе. В данном случае ссылка будет иметь вид \citemy{...}.
\newcites{my}{Список публикаций}

% Путь к файлам с иллюстрациями
\graphicspath{{../../images/}}

\begin{document}
% Включение файла с общим текстом диссертации и автореферата
% (текст титульного листа и характеристика работы).
% Общие поля титульного листа диссертации и автореферата
\institution{Федеральное государственное учреждение\\
	<<Федеральный исследовательский центр <<Информатика и управление>>
	Российской академии наук>>}

\topic{Методы эффективного решения комбинаторных задач на основе знакового представления знаний}

\author{Панов Александр Игоревич}

\specnum{05.13.17}
\spec{Теоретические основы информатики}
%\specsndnum{01.04.07}
%\specsnd{Физика конденсированного состояния}

\scon{Осипов Геннадий Семенович}
\sconstatus{д.~ф.-м.~н., проф.}
%\sconsnd{ФИО второго консультанта}
%\sconsndstatus{д.~ф.-м.~н., проф.}

\city{Москва}
\date{\number\year}

% Общие разделы автореферата и диссертации
\mkcommonsect{actuality}{Актуальность темы исследования.}{%
Текст об актуальности. Ссылка~\cite{Osipov2002a}.
}

\mkcommonsect{development}{Степень разработанности темы исследования.}{
Текст о степени разработанности темы.
}

\mkcommonsect{objective}{Цели и задачи диссертационной работы:}{%
Список целей.

Для достижения поставленных целей были решены следующие задачи:
}

\mkcommonsect{novelty}{Научная новизна.}{%
Текст о новизне.
}

\mkcommonsect{value}{Теоретическая и практическая значимость.}{%
Результаты, изложенные в диссертации, могут быть использованы для ...
}

\mkcommonsect{methods}{Методология и методы исследования.}{%
Текст о методах исследования.
}

\mkcommonsect{results}{Положения, выносимые на защиту:}{%
Текст о положениях и результатах.
}

\mkcommonsect{approbation}{Степень достоверности и апробация результатов.}{%
Основные результаты диссертации докладывались на следующих конференциях:
}

\mkcommonsect{pub}{Публикации.}{%
Материалы диссертации опубликованы в $N$ печатных работах, из них $n_1$
статей в рецензируемых журналах~\citemy{Panov2014a,Panov2014b,Panov2013a}, $n_2$ статей в
сборниках трудов конференций и $n_3$ тезисов докладов.
}

\mkcommonsect{contrib}{Личный вклад автора.}{%
Содержание диссертации и основные положения, выносимые на защиту, отражают персональный вклад автора в опубликованные работы.
Подготовка к публикации полученных результатов проводилась совместно с соавторами, причем вклад диссертанта был определяющим. Все представленные в диссертации результаты получены лично автором.
}

\mkcommonsect{struct}{Структура и объем диссертации.}{%
Диссертация состоит из введения, обзора литературы, $n$ глав, заключения и библиографии.
Общий объем диссертации $P$ страниц, из них $p_1$ страницы текста, включая $f$ рисунков.
Библиография включает $B$ наименований на $p_2$ страницах.
}

% номер копии для грифа секретности
%\copynum{1}
% класс доступа
%\classlabel{Для служебного пользования}

\title{АВТОРЕФЕРАТ\\
диссертации на соискание ученой степени\\
кандидата физико-математических наук}

\maketitle

% Внутренняя сторона обложки
\thispagestyle{empty}
\vspace*{-2cm}
\noindent
\begin{center}
Работа выполнена в \emph{название организации}.
\end{center}
\vskip1ex
\begin{tabularx}{\linewidth}{lp{1cm}X}
Научный руководитель:  & & \emph{доктор физико-математических наук}, \\
                       & & \emph{профессор}, \\
                       & & \emph{Осипов Геннадий Семёнович}
\\
Официальные оппоненты: & & \emph{доктор физико-математических наук}, \\
                       & & \emph{профессор}, \\
                       & & \emph{Редько Владимир Георгиевич}\\
                       & & \emph{доктор физико-математических наук}, \\
                       & & \emph{профессор}, \\
                       & & \emph{Кузнецов Сергей Олегович}
\\
Ведущая организация:   & & \emph{Федеральное государственное бюджетное учреждение науки Институт проблем управления им. В.\,А.~Трапезникова Российской академии наук}\\
\end{tabularx}

\vskip2ex\noindent
Защита состоится \datefield{21 мая 2015~г.} в 13 часов 00 минут
на заседании диссертационного совета \emph{шифр совета} при \emph{название
организации, при которой создан совет}, расположенном по адресу:
\emph{адрес}

\vskip1ex\noindent
С диссертацией можно ознакомиться в библиотеке
\emph{название организации}.

\vskip1ex\noindent
Автореферат разослан \datefield{8 апреля 2015~г.}

\vskip2ex\noindent
Отзывы и замечания по автореферату в двух экземплярах, заверенные
печатью, просьба высылать по вышеуказанному адресу на имя ученого секретаря
диссертационного совета.

\vfill\noindent
Ученый секретарь\\
диссертационного совета,\\
\emph{доктор физико-математических наук}%
\hfill
\makeatletter
% вставка файла, содержащего факсимиле ученого секретаря
\ifDis@facsimile
  \raisebox{-4pt}{\includegraphics[width=3cm]{sec-facsimile}}\hfill
\fi%
\makeatother%
\emph{Рязанов~В.\,В.}

\clearpage

\nsection{Общая характеристика работы}

% Актуальность работы
\actualitysection
\actualitytext

% Степень разработанности темы исследования
\developmentsection
\developmenttext

% Цели и задачи диссертационной работы
\objectivesection
\objectivetext

% Научная новизна
\noveltysection
\noveltytext

% Теоретическая и практическая значимость
\valuesection
\valuetext

% Методология и методы исследования
\methodssection
\methodstext

% Положения, выносимые на защиту
\resultssection
\resultstext

% Степень достоверности и апробация результатов
\approbationsection
\approbationtext

% Публикации
\pubsection
\pubtext

% Личный вклад автора
\contribsection
\contribtext

% Структура и объем диссертации
\structsection
\structtext

\nsection{Содержание работы}

\textbf{Во Введении} обоснована актуальность диссертационной работы,
сформулирована цель и аргументирована научная новизна исследований, показана
практическая значимость полученных результатов, представлены выносимые на
защиту научные положения.

\textbf{В первой главе} ...

Содержание первой главы.

\textbf{Во второй главе} ...

Содержание второй главы.


\textbf{В третьей главе} ...

Содержание третьей главы.


\textbf{В Заключении}

% ----------------------------------------------------------------
\renewcommand\bibsection{\nsection{Список публикаций}}

\bibliographystylemy{gost705}
\bibliographymy{synopsis}

\renewcommand\bibsection{\nsection{Цитированная литература}}

\bibliographystyle{gost705}
\bibliography{synopsis}
% ----------------------------------------------------------------
% Выходные данные
\clearpage
\thispagestyle{empty}
\normalfont\selectfont
\vspace*{2cm}
\begin{center}
\textit{Научное издание}\\
\vskip 2cm
\makeatletter
\@author
\vskip 1.5cm
\@title{} на тему:\\
\@topic\\
\makeatother
\end{center}
\vfill
Подписано в печать~25.01.2011.
Формат~$60 \times 90$~1/16.
Тираж~100~экз.
Заказ~256.\\[2ex]
\noindent
Санкт-Петербургская издательская фирма <<Наука>> РАН.
199034, Санкт-Петербург, Менделеевская линия, 1,
\href{http://www.naukaspb.spb.ru}{http://www.naukaspb.spb.ru}

\end{document}
