%%% Макет страницы %%%
\oddsidemargin=-13pt
\topmargin=-66pt
\headheight=12pt
\headsep=38pt
\textheight=732pt
\textwidth=484pt
\marginparsep=14pt
\marginparwidth=43pt
\footskip=14pt
\marginparpush=7pt
\hoffset=0pt
\voffset=0pt
%\paperwidth=597pt
%\paperheight=845pt
\parindent=1cm %размер табуляции (для красной строки) в начале каждого абзаца
\hyphenpenalty=10000		% Запрещаем переносы во всем тексте

\renewcommand{\baselinestretch}{1.25}
\newfloat{scheme}{tb}{sch}

%%% Общая информация %%%
\author{Панов~А.\,И.} % Фамилия И.О. автора

%%% Кодировки и шрифты %%%
\renewcommand{\rmdefault}{ftm} % Включаем Times New Roman

%%% Выравнивание и переносы %%%
\sloppy
\clubpenalty=10000
\widowpenalty=10000

%%% Библиография %%%
\makeatletter
\bibliographystyle{ugost2008p}
\renewcommand{\@biblabel}[1]{#1.}	% Заменяем библиографию с квадратных скобок на точку:
\makeatother

%%% Изображения %%%
\graphicspath{{../../images/}} % Пути к изображениям

%%% Теормоподобные окружения %%%
\theoremstyle{plain}
\newtheorem{Theorem}{Теорема}
\newtheorem{Pred}{Утверждение}
\newtheorem{Def}{Определение}

\floatname{algorithm}{Алгоритм}
\algrenewcommand\algorithmicrequire{\textbf{Вход:}}
\algrenewcommand\algorithmicensure{\textbf{Выход:}}
\algrenewcommand\algorithmicforall{\textbf{для всех}}
\algrenewcommand\algorithmicwhile{\textbf{пока}}
\algrenewcommand\algorithmicif{\textbf{если}}
\algrenewcommand\algorithmicthen{\textbf{то}}
\algrenewcommand\algorithmicelse{\textbf{иначе}}
\algrenewcommand\algorithmicreturn{\textbf{вернуть}}
\algrenewcommand\algorithmicdo{}
\renewcommand{\algorithmiccomment}[1]{{\quad\sl // #1}}

\DeclareMathOperator*{\argmax}{arg\,max}

\def\BibSection#1#2{\subsection*{#2}}
