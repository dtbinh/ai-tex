%%% Поля и разметка страницы %%%
\usepackage{geometry} % Для последующего задания полей

%%% Кодировки и шрифты %%%
\usepackage{cmap} 							% Улучшенный поиск русских слов в полученном pdf-файле
\usepackage[T2A]{fontenc} 					% Поддержка русских букв
\usepackage[utf8]{inputenc} 				% Кодировка utf8
\usepackage[english, russian]{babel} 		% Языки: русский, английский
\usepackage{pscyr} % Нормальные шрифты
%\usepackage[style=verbose-ibid,backend=bibtex]{biblatex}

%%% Математические пакеты %%%
\usepackage{amsthm,amsfonts,amsmath,amssymb,amscd} % Математические дополнения от AMS

%%% Оформление абзацев %%%
\usepackage{indentfirst} % Красная строка

%%% Таблицы %%%
\usepackage{makecell} % Улучшенное форматирование таблиц

%%% Общее форматирование
\usepackage[singlelinecheck=off,center]{caption} % Многострочные подписи
\usepackage{soul} % Поддержка переносоустойчивых подчёркиваний и зачёркиваний

%%% Библиография %%%
\usepackage{cite}

%%% Гиперссылки %%%
\usepackage[plainpages=false,pdfpagelabels=false]{hyperref}

\usepackage{subcaption}					% Для вложенных изображений

%%% Изображения %%%
\usepackage{graphicx} % Подключаем пакет работы с графикой
\usepackage[ruled]{algorithm}					% Пакеты листингов и алгоритмов
\usepackage[noend]{algpseudocode}

\usepackage{titlesec}					% Форматирование заголовков